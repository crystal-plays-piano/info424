\documentclass[12pt,letterpaper]{article}

% - package setup
\usepackage[doublespacing]{setspace}
\usepackage{csvsimple}

% - title setup
\title{Assignment 8}
\author{Crystal Mandal}
\date{last edited \today}

\usepackage{graphics}
\begin{document}

\maketitle

\section{Question 1:}
The music (40 sounds) generated with the requisite parameters is displayed below:

\begin{quote}
{%
\parindent 0pt
\noindent
\ifx\preLilyPondExample \undefined
\else
  \expandafter\preLilyPondExample
\fi
\def\lilypondbook{}%
\includegraphics{d2/lily-e992c0ca-1}%
\ifx\betweenLilyPondSystem \undefined
  \linebreak
\else
  \expandafter\betweenLilyPondSystem{1}%
\fi
\includegraphics{d2/lily-e992c0ca-2}%
\ifx\betweenLilyPondSystem \undefined
  \linebreak
\else
  \expandafter\betweenLilyPondSystem{2}%
\fi
\includegraphics{d2/lily-e992c0ca-3}%
\ifx\betweenLilyPondSystem \undefined
  \linebreak
\else
  \expandafter\betweenLilyPondSystem{3}%
\fi
\includegraphics{d2/lily-e992c0ca-4}%
% eof
%
\ifx\postLilyPondExample \undefined
\else
  \expandafter\postLilyPondExample
\fi
}
\end{quote}

\section{Question 2:}
\subsection{Markov Bank}
The bank of pitches (first 7 sounds) used in markov chain generation is:

\begin{quote}
{%
\parindent 0pt
\noindent
\ifx\preLilyPondExample \undefined
\else
  \expandafter\preLilyPondExample
\fi
\def\lilypondbook{}%
\includegraphics{5a/lily-c71f8e42-1}%
% eof
%
\ifx\postLilyPondExample \undefined
\else
  \expandafter\postLilyPondExample
\fi
}
\end{quote}

\subsection{Markov Chain Table}
The table generated for the Markov Chain is:

\csvautotabular{../../python/ass8-markov-table.csv}

\subsection{Markov Chain}
The music (20 sounds) generated with the requisite parameters is displayed below:

\begin{quote}
{%
\parindent 0pt
\noindent
\ifx\preLilyPondExample \undefined
\else
  \expandafter\preLilyPondExample
\fi
\def\lilypondbook{}%
\includegraphics{1d/lily-912972b4-1}%
\ifx\betweenLilyPondSystem \undefined
  \linebreak
\else
  \expandafter\betweenLilyPondSystem{1}%
\fi
\includegraphics{1d/lily-912972b4-2}%
% eof
%
\ifx\postLilyPondExample \undefined
\else
  \expandafter\postLilyPondExample
\fi
}
\end{quote}

\section{Methodology}
The music is generated by the `assignment8.py' script in


\end{document}


